\subsubsection{Numerical confirmation of the theory}

Let us consider a numerical experiment on a test case in order to clarify the theory presented above. Let us consider the domain $D = [-1, 1]^2$ endowed with mixed reflecting and killing boundary conditions and the deterministic transport field given by
\begin{equation*}
	f(x, y) = \begin{pmatrix} |x|, & y \end{pmatrix}^T.
\end{equation*}
In the domain, the transport field is Lipschitz continuous with constant $K = 1$. Furthermore, let us consider $\sigma = 1$ and the time final time to be $T = 10$. The perturbed transport field $f^\epl$ is given by a piecewise constant interpolation of $f$ on a structured grid of equal spacing $\epl$ in the two directions. We estimate the mean exit time $\bar\tau$ from the domain by means of a Montecarlo simulation exploiting CEM performed both on the perturbed and the non-perturbed equations. We fix $h = T / 2^{10}$ and we consider $M = 10000$ realizations of the numerical solution in order to make both the statistical error and the error relative to the numerical integration negligible. We perform the simulation for eight different values $\epl_i$ given by $\epl_i = 2^{-i}, i = 1,\dots, 8$. Numerical results (Figure \ref{fig:ConvTheory}) show a convergence of first order, thus confirming the theoretical bound given by \eqref{eq:DiffXeplnXn}. Let us remark that the order of convergence with respect to $\epl$ is preserved even though the theoretical results we presented concern the value of the solution while in this numerical experiment we consider the mean exit time.

\begin{figure}[t]
    \centering
    \resizebox{0.6\linewidth}{!}{\begin{frame}
\frametitle{Outline of the presentation}
\begin{itemize}
	\item \color{mygray} Expected exit time from a domain
	\begin{itemize} \color{mygray}
		\item[--] Setting
		\item[--] Numerical methods
		\item[--] Numerical experiments
	\end{itemize} \color{black}
	\item Theoretical investigation: perturbed SDE's 
	\item The uncertain Darcy's problem 
\end{itemize}
\end{frame}

\begin{frame}
\frametitle{Theoretical investigation - Motivation}
Consider the velocity field $\uDarcy$ solution of the Darcy's problem and an approximation $\color{blue} \tilde u$ of $\uDarcy$. Consider then the SDE
\begin{equation*}
	\left \{
	\begin{aligned}
		dX(t) &= \uDarcy (X(t)) \color{black} dt + \sigma dW(t), && 0 \leq t \leq T, \\
		X(0) &= X_0 \in D, \\
	\end{aligned} \right.
\end{equation*}
and the SDE
\begin{equation*}
	\left \{
	\begin{aligned}
		d\tilde X(t) &= \color{blue} \tilde u(\tilde X(t)) \color{black} dt + \sigma dW(t), && 0 \leq t \leq T, \\
		\tilde X(0) &= X_0 \in D, \\
	\end{aligned} \right.
\end{equation*}
\underline{Problem}. If $\color{blue} \tilde u$ converges to $\uDarcy$, does $\tilde X(t)$ converge to $X(t)$? What is the effect on the order of convergence of numerical methods?
\end{frame}

\begin{frame}
\frametitle{Theoretical investigation - Convergence of the solution}
In an abstract form, consider $f\colon \R \to \R$ and the SDE
\begin{equation*}
\left \{
\begin{aligned}
	dX(t) &= f(X(t))dt + \sigma dW(t), && 0 < t \leq T, \\
	X(0) &= X_0,
\end{aligned} \right .
\end{equation*}
Consider a perturbation of the transport field $f^\epl\colon \R \rightarrow \R$ and
\begin{equation*}
\left \{
\begin{aligned}
	dX^\epl(t) &= f^\epl(X^\epl(t))dt + \sigma dW(t), && 0 < t \leq T, \\
	X^\epl(0) &= X_0.
\end{aligned} \right .
\end{equation*}

\end{frame}

\begin{frame}
\frametitle{Theoretical investigation - Convergence of the solution}
\begin{proposition} If the following assumptions are verified for a constant $K > 0$
\begin{enumerate}
	\item $|f(x) - f(y)| \leq K|x - y|, \: \forall x,y \in \R$,
	\item $|f(x)| \leq K(1 + |x|), \: \forall x \in \R$,
\end{enumerate}
and if the solution $X^\epl(t)$ of the perturbed SDE exists, then
\begin{equation*}
	\E \sup_{0 \leq t \leq T} |X^\epl(t) - X(t)|^2 \leq  2T^2 \|f - f^\epl\|_{\infty}^2 e^{2K^2T^2}.
\end{equation*}
\end{proposition}
\underline{Remark}. We proved similar results for two independent Brownian motions $W_1, W_2$ and in the $d$-dimensional case.
\end{frame}

\begin{frame}
\frametitle{Theoretical investigation - Numerical convergence}
Consider the Euler-Maruyama method applied to the perturbed SDE
\begin{equation*}
\left \{
\begin{aligned}
	X^\epl_{n+1} &= X^\epl_n + f^\epl(X^\epl_n)h + \sigma (W(t_{n+1}) - W(t_n)), && n = 0, \dots, N - 1, \\
	X^\epl_0 &= X_0.
\end{aligned} \right .
\end{equation*}
\underline{Problem}. Determine the convergence of $X^\epl_n$ to $X(t)$ with respect to $h$ and $\epl$.
\end{frame}

\begin{frame}
\frametitle{Theoretical investigation - Numerical convergence}
\begin{proposition} If the following assumptions are verified for a constant $K > 0$
\begin{enumerate}
	\item $|f(x) - f(y)| \leq K|x - y|, \: \forall x,y \in \R$,
	\item $|f(x)| \leq K(1 + |x|), \: \forall x \in \R$,
\end{enumerate}
and if the solution $X^\epl(t)$ of the perturbed SDE exists, then
\begin{equation*}
	\sup_{n = 0, \dots, N} \E \|X(nh) - X^\epl_n\| \leq C h +  \|f^\epl - f\|_{\infty} \frac{e^{KT} - 1}{K}, 
\end{equation*}
with $C$ a real constant independent of $h$ and depending only on the final time $T$ and the Lipschitz constant $K$ of $f$.
\end{proposition}
\underline{Idea of the proof}. Use triangular inequality summing and subtracting the variable $X_n$ given by Euler-Maruyama applied to the non-perturbed equation.
\end{frame}

\begin{frame}
\frametitle{Theoretical investigation - Numerical convergence}
\underline{Remark}. If $D$ is a square domain and $f^\epl$ is a piece-wise constant interpolation of $f$ on a regular grid of equal size $\epl$ in the two directions
\begin{equation*}
	\sup_{n = 0, \dots, N} \E \|X(nh) - X^\epl_n\| = O(h) + O(\epl).
\end{equation*}
Therefore, set $h = O(\epl)$ to avoid extra computational cost. 

\vspace{0.5cm}
\underline{Numerical experiments confirm this behaviour.}

\end{frame}
 }  
    \caption{Convergence with respect to the interpolation characteristic size $\epl$ for a test case.}
    \label{fig:ConvTheory}
\end{figure}
