\subsection{Numerical experiments}

We perform a numerical experiment in order to verify the theoretical bounds presented above. Let us consider the domain $D = [-1, 1]^2$ and the deterministic transport field given by
\begin{equation*}
	f(x, y) = \frac{1}{2} \begin{pmatrix} x^2 + y^2, & |x - y| \end{pmatrix}^T.
\end{equation*}
In the domain, the transport field is Lipschitz continuous. Our aim is verifying whether the numerical solution computed using a piecewise constant interpolation of $f$ leads to convergence with respect to the characteristic size of the grid used for interpolation. In the following, $f^\epl$ denotes the piecewise constant interpolation over $D$ of $f$ on a structured grid of equal size $\epl$ in both directions. Since the theoretical results interest the value of the solution itself and not the mean exit time from a domain, we start by considering the value itself. Then, we verify if the results are valid for the exit time as well.

\vspace{2mm}
\noindent \textbf{Numerical solution.} We consider the boundary conditions to be reflecting on all the boundary of $D$. We fix the parameters to be $T = 1, \sigma = 1, X_0 = (0, 0)^T$ and perform a Montecarlo simulation over $M = 10^4$ realizations of Euler-Maruyama. Let us remark that since the boundary is completely reflecting, there is no distinction between CEM and DEM. Maintaining the notation of Proposition \ref{thm:StrongConv}, we compute the weak error of $X_n^\epl$ with respect to $X_n$, which should be of order $O(\epl)$. The reference solution $X_n$ is computed over $N = 2^{10}$ timesteps over the time span. Then, we vary $\epl$ in the range $\epl_i = 2^{-i}, i = 1,\dots, 8$ and compute the numerical solution $X_n^\epl$ either fixing the number of timesteps $N = 2^9$ or varying it so that $h_i = \epl_i$, \textit{i.e.}, the time and space discretizations have the same order of magnitude. In the first case, we wish to overkill the error due to numerical integration, while in the second case we wish to maintain an acceptable computational cost. Moreover, the bound presented in Remark \ref{rmk:InterpNum} implies that theoretically the error should be independent of $h$, so we would expect similar results for both the approaches. Results (Figure \ref{fig:TheoryX}) confirm the theoretical results, with a clear rate of convergence equal to one, and the errors which are similar for the two approaches.

\vspace{2mm}
\noindent \textbf{Mean exit time.} We perform an experiment with the same values for all parameters as above, but we consider mixed killing and reflecting boundary conditions and we estimate the exit time $\tau$. We wish that the theoretical results obtained for the solution apply practically to the exit time itself. We use the same strategies as above, either fixing a small step size $h$ for any value of $\epl$, or maintaining $h$ equal to $\epl$. Results (Figure \ref{fig:TheoryTau}) show that the error is $O(\epl)$ in both cases, with a smaller constant in case a small value of $h$ is chosen. On the other side, we notice that fixing, \textit{e.g.}, the error to $0.01$, the computational time is in this case approximately sixteen times smaller in case $h$ and $\epl$ are balanced.

\begin{figure}[t]
    \centering
    \begin{subfigure}{0.49\linewidth}
        \centering
        \resizebox{1\linewidth}{!}{\input{Darcy/Pictures/TheoryX.tikz} }   
        \caption{Error on the solution.}
        \label{fig:TheoryX}
    \end{subfigure}
    \begin{subfigure}{0.49\linewidth}
        \centering
        \resizebox{1\linewidth}{!}{\input{Darcy/Pictures/TheoryTau.tikz} }  
        \caption{Error on the exit time.}
        \label{fig:TheoryTau}
    \end{subfigure}    
    \caption{Convergence of the numerical solution with respect to the interpolation characteristic size $\epl$.}
    \label{fig:Theory}
\end{figure}
