\subsection{Numerical Experiment - Convergence of the method}

The method we presented above can lead to incorrect solutions if one of the two parameters $\Delta_A$ and $h$, the timestep chosen for the integration of the SDE \eqref{eq:GeneralDarcySDE} with DEM or CEM, is not small enough. Therefore, we perform numerical experiments in order to have an inspect on the convergence of the expected exit time from the domain $D$. Let us consider the equation \eqref{eq:DarcyProblem}, with parameters $\sigma_A = 1, \nu = 0.5, L_c = 0.05$ for \eqref{eq:CovFunction}. We generate the random field $A$ and find the pressure and velocity fields using the FEM solver \texttt{FreeFem++}. We consider the value $\Delta_A$ to be an input parameter for our simulation. Then, we interpolate the retrieved velocity feld as descripted above, and perform a Montecarlo simulation to get a numerical expectation of $\tau$. In order to verify convergence, we vary separately the two parameters of interest, \textit{i.e.}, we will consider $h$ and $\Delta_A$ to be fixed alternatively. Let us remark that in order to have a consistent comparison, the values of the exit time have to be estimated using the same realization of $A$, restricted on a coarser grid if the scope is testing a larger value of $\Delta_A$. 

\noindent \textbf{First experiment.} Let us consider the final time $T = 10$ and $\sigma = 0.3$ in equation \eqref{eq:GeneralDarcySDE}, and a Montecarlo simulation over $M = 100000$ trajectories. In Figure \ref{fig:ConvDeltaAh}, the values of the expected exit time for $\Delta_A = 0.5, 0.0625, 0.0078, h_i = T/N_i, N_i = 2^i, i = 0, 1, \dots, 7$ are displayed. It is possible to notice that the results for the fine and the medium grids are approximatively equal, while for the coarse grid the expected exit time do not converge to the same value. 

\begin{figure}[t]
    \centering
    \resizebox{0.8\linewidth}{!}{\input{Darcy/Pictures/Coarsening.tikz} }  
    \caption{Results for three values of $\Delta_A$ (coarse, medium, fine) and convergence with respect to $h$ of the mean exit time.}
    \label{fig:ConvDeltaAh}
\end{figure}
