\section{Darcy problem}
The two methods for approximating the mean exit time have been investigated in a general frame. In the following we will consider \ref{eq:GeneralModel} with $f\colon \mathbb{R}^2 \rightarrow \mathbb{R}^2$ given by the solution of the uncertain Darcy problem. 

\subsection{Problem statement}
Let us consider a domain $D \subset \mathbb{R}^2$. Let us define the Neumann boundaries of $D$ as $\Gamma_N$, its inlet boundary as $\Gamma_{in}$ and its outlet boundary as $\Gamma_{out}$. Then, we search the solution of the following problem
\begin{equation}
	\label{eq:DarcyProblem}
	\begin{cases}
		u = -A \nabla p, & \text{in } D, \\
		\nabla\cdot u = 0, & \text{in } D, \\
		p = p_0, & \text{on } \Gamma_{in},\\
		p = 0, & \text{on } \Gamma_{out}, \\
		\nabla p = 0, & \text{on } \Gamma_N,
	\end{cases}
\end{equation}
where $A$ is a random field. The equation is solved using Finite Elements, thus retrieving an approximate solution $u_h$ of the velocity field. Then, given $\sigma \in \mathbb{R}$ we consider the following SDE in the same domain $D$
\begin{equation}
	\label{eq:GeneralDarcySDE}
	\begin{cases}
		dX(t) = u_h(X) dt + \sigma dW(t), & 0 < t \leq T, \\
		X(0) = X_0, X_0 \in D,
	\end{cases}
\end{equation}
where we set the boundary conditions to be reflecting on $\Gamma_N$ and killing on both $\Gamma_{in},\Gamma_{out}$.

\begin{figure}[t]
    \centering
    \begin{subfigure}{0.49\linewidth}
        \centering
        \includegraphics [width=1\linewidth]{Darcy/Pictures/A.jpg}
        \caption{Random field.}
        \label{fig:DarcyA}
    \end{subfigure}
    \begin{subfigure}{0.49\linewidth}
        \centering
        \includegraphics [width=1\linewidth]{Darcy/Pictures/P.jpg}
        \caption{Pressure field.}
        \label{fig:DarcyP}
    \end{subfigure}    
    \begin{subfigure}{0.49\linewidth}
        \centering
        \includegraphics [width=1\linewidth]{Darcy/Pictures/Ux.jpg}
        \caption{$x$ component of velocity field.}
        \label{fig:DarcyUx}
    \end{subfigure}
    \begin{subfigure}{0.49\linewidth}
        \centering
        \includegraphics [width=1\linewidth]{Darcy/Pictures/Uy.jpg}
        \caption{$y$ component of velocity field.}
        \label{fig:DarcyUy}
    \end{subfigure}    
    \caption{Approximate solution of the uncertain Darcy problem.}
    \label{fig:DarcyResults}
\end{figure}



\subsection{Finite Elements solution of the Darcy problem}

Let us consider the domain $D = [-1,1]^2$. The random field $A$ in \eqref{eq:DarcyProblem} is chosen to be lognormal, \textit{i.e.}, 
\begin{equation}\label{eq:RandomField}
	A = e^\gamma,
\end{equation}
where $\gamma$ is a normal random field defined by its covariance function $\mathrm{cov}_\gamma(x_1,x_2)$ for any couple of points $x_1,x_2$ in the domain $D$. The covariance function is of the Matern family \cite{Nobile2015}, thus having the following form
\begin{equation}\label{eq:CovFunction}
	\mathrm{cov}_\gamma(x_1,x_2) = \frac{\sigma_A^2}{\Gamma(\nu)2^{\nu-1}}\Big(\sqrt{2\nu}\frac{|x_1-x_2|}{L_c}\Big)^\nu K_{\nu}\Big(\sqrt{2\nu}\frac{|x_1-x_2|}{L_c}\Big), \quad \nu \geq 0.5,
\end{equation}
where $\sigma_A^2$ is the variance, $L_c$ is the correlation length, $\Gamma$ is the gamma function, $K_\nu$ is the modified Bessel function of the second kind and $\nu$ is a parameter. Let us remark that the covariance function does not depend on $x_1, x_2$ but only on their euclidean distance $|x_1 - x_2|$. The regularity of the covariance function and of the realizations of $A$ depend on $\nu$. In particular each realization of the field $A$ is $\alpha$-Hölder continuous for $0 < \alpha < \nu$. Therefore, for $\nu = 1 + \epl, \epl > 0$, the random field is Lipschitz continuous. Results concerning further regularity properties of $A$ can be found in \cite{Nobile2015}. The realizations of $A$ are computed using a discrete Fourier transformation on the vertices of a grid of $D$, equispaced on both the $x$ and $y$ directions with the same spacing $\Delta_A$. Then, the numerical solution $\hat{p}$ of \eqref{eq:DarcyProblem} is obtained with linear Finite Elements on a regular mesh $T_p$ with maximum element size $\Delta_p$. Since the vertices of the grid on which we compute $A$ do not coincide with the vertices of $T_p$, we interpolate $A$ on $T_p$. Then, the velocity field $\hat{u}$ is retrived computing the gradient of $\hat{p}$ as in equation \eqref{eq:DarcyProblem}. We choose to perform this procedure using the scientific software \texttt{FreeFem++}. The results for a realization of $A$ are shown in Figure \ref{fig:DarcyResults}, where the value of the inlet pressure $p_0$ is equal to 1, and the parameters for the random field are $\nu = 0.5, L_c = 0.05$. 


\input{Darcy/SDE}
