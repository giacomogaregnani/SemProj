\subsection{Theoretical investigation}
\subsubsection{Analysis of analytical convergence}
The procedure explained above could give rise to theoretical issues concerning the existence and uniqueness of the solution of \eqref{eq:GeneralDarcySDE}. In particular, when performing step 3. of the procedure as in the list above, the hypothesis that the transport field of the SDE is Lipschitz continuous is not valid anymore, and therefore we have no guarantee that \eqref{eq:GeneralDarcySDE} is well-posed. On the other side, we can hope that if $\tilde{u}$ tends uniformly to the exact solution $u$ of \eqref{eq:DarcyProblem}, provided that the latter is a smooth function, the solution of the SDE that has $\tilde{u}$ as a transport field tends to the solution given by $u$. In the following we present a result that supports our numerical procedure. Let us consider $\sigma \in \R, f\colon \R^d \rightarrow \R^d$, $W_1(t)$ a standard $d$-dimensional Wiener process and the following SDE
\begin{equation}\label{eq:AppSDE}
\left \{
\begin{aligned}
	dX(t) &= f(X(t))dt + \sigma IdW_1(t), && 0 < t \leq T, \\
	X(0) &= X_0,
\end{aligned} \right .
\end{equation}
where $I$ is the identity matrix in $\R^{d\times d}$. Moreover, let us consider a perturbation of the transport field $f^\epl\colon \R^d \rightarrow \R^d$ and a standard $d$-dimensional Wiener process $W_2(t)$ not necessarily independent of $W_1(t)$. Then, we consider the perturbed SDE 
\begin{equation}\label{eq:AppSDEPer}
\left \{
\begin{aligned}
	dX^\epl(t) &= f^\epl(X^\epl(t))dt + \sigma I dW_2(t), && 0 < t \leq T, \\
	X^\epl(0) &= X_0.
\end{aligned} \right .
\end{equation}
We can state the following preliminary result
\begin{lemma}\label{lem:Lemma1} With the notation above, let us consider $d = 1$, $W_1 = W_2 = W$ almost everywhere. Then, given a real constant $K$ and with the following assumptions
\begin{enumerate}
	\item $|f(x) - f(y)| \leq K|x - y|, \: \forall x,y \in \R$
 	\item $\sup_{x\in \R}|f(x) - f^\epl(x)| \stackrel{\epl \rightarrow 0}\longrightarrow 0$,
\end{enumerate}
the solution $X^\epl(t)$ of \eqref{eq:AppSDEPer} converges strongly to the solution $X(t)$ of \eqref{eq:AppSDE} in $L^2(\Omega)$, \textit{i.e.}
\begin{equation*}
	\E \sup_{0 \leq t \leq T} |X^\epl(t) - X(t)|^2 \stackrel{\epl \rightarrow 0}\longrightarrow 0.
\end{equation*}
\end{lemma}

\begin{proof}
For almost all $\omega$
\begin{equation*}
\begin{aligned}
	|X^\epl(t) - X(t)|^2  &= \Big|\int_0^t (f^\epl (X^\epl(s)) - f(X(s)))ds \Big|^2 \\
	&\leq T \int_0^t |f^\epl (X^\epl(s)) - f(X(s))|^2 ds   && \{\text{Cauchy-Schwarz inequality}\}\\
	&\leq 2T \int_0^t |f^\epl (X^\epl(s)) - f(X^\epl(s))|^2 ds \\
	&\quad + 2T \int_0^t |f (X^\epl(s)) - f(X(s))|^2 ds && \{\text{Triangular inequality, Young inequality}\} \\
	&\leq 2T^2 \sup_{x\in \R}|f(x) - f^\epl(x)|^2 \\
	&\quad + 2T^2K^2 \int_0^t |X^\epl(s) - X(s)|^2ds. && \{\text{Assumption 2. and 1.}\} 
\end{aligned} 
\end{equation*}
We then apply Gronwall's inequality, which gives 
\begin{equation}\label{eq:TheoryEpsilon}
	|X^\epl(t) - X(t)|^2 \leq  2T^2 \sup_{x\in \R}|f(x) - f^\epl(x)|^2 e^{2K^2T^2}, \quad a.e.
\end{equation}
Since the right hand side of the inequality is constant, we can then take the supremum over time and the expectation at both sides, \textit{i.e.},
\begin{equation*}
	\E \sup_{0 \leq t \leq T} |X^\epl(t) - X(t)|^2 \leq  2T^2 \sup_{x\in \R}|f(x) - f^\epl(x)|^2 e^{2K^2T^2}.
\end{equation*}
Then, assumption 2. gives the result.
\end{proof}
\noindent We now consider the case of two independent Wiener processes.

\begin{lemma}\label{lem:Lemma2} With the notation above and the assumptions 1. and 2. of Lemma \ref{lem:Lemma1}, let us consider $W_1$ independent of $W_2$ and $d = 1$. Then
\begin{equation*}	
	\E \sup_{0 \leq t \leq T} |X^\epl(t) - X(t)|^2 \leq 4T(T \sup_{x\in \R}|f(x) - f^\epl(x)|^2 + 4\sigma^2) e^{2K^2T^2}.
\end{equation*}
\end{lemma}

\begin{proof} Let us compute the difference between $X^\epl(t)$ and $X(t)$. Applying Young's inequality, we get
\begin{equation*}
\begin{aligned}
	\E \sup_{0 \leq t \leq T} |X^\epl(t) - X(t)|^2 &\leq 2\E \sup_{0 \leq t \leq T}\Big|\int_0^t (f^\epl (X^\epl(s)) - f(X(s)))ds \Big|^2 \\
		&\quad + 2\sigma^2 \E \sup_{0 \leq t \leq T} \Big|\int_0^t dW_1(s) - \int_0^t dW_2(s)  \Big|^2.
\end{aligned}
\end{equation*}
For the first term, we can apply Lemma \ref{lem:Lemma1}. Let us define $Z(t) := W_1(t) - W_2(t)$. $Z(t)$ is a standard Wiener process with variance $2t$, thus it is a martingale. Hence, we can apply Doob's martingale inequality (\textit{e.g.}, \cite{Wang1991}) to the second term, obtaining
\begin{equation*}
	\E \sup_{0 \leq t \leq T} \Big|\int_0^t dW_1(s) - \int_0^t dW_2(s)  \Big|^2 = \E \sup_{0 \leq t \leq T} | Z(t) |^2 \leq 4 \E |Z(T)|^2 = 8T.
\end{equation*}
Therefore, we get the result
\begin{equation*}
	\E \sup_{0 \leq t \leq T} |X^\epl(t) - X(t)|^2 \leq (4T^2 \sup_{x\in \R}|f(x) - f^\epl(x)|^2 + 16T\sigma^2) e^{2K^2T^2}.
\end{equation*}
\end{proof}
\noindent With those preliminary results, we can now consider the general case of a $d$-dimensional SDE.
\begin{theorem}\label{th:Prop} With the notation above and if $f\colon \R^d \rightarrow \R^d, f^\epl\colon \R^d \rightarrow \R^d,$ satisfy
\begin{enumerate}
	\item $\|f(x) - f(y)\| \leq K\|x - y\|_2, \: \forall x,y \in \R^d$,
 	\item $\sup_{x\in \R}|f_i(x) - f_i^\epl(x)| \stackrel{\epl \rightarrow 0}\longrightarrow 0, \: i = 1, \dots, d$,
\end{enumerate}
then it is true for $X(t), X^\epl(t)$ the solutions of \eqref{eq:AppSDE} and \eqref{eq:AppSDEPer} that 
\begin{equation}\label{eq:Proposition}
	\E \| X^\epl(t) - X(t) \|_2^2 \leq \Big(4T^2 \sum_{i=1}^d \sup_{x\in \R}|f_i(x) - f_i^\epl(x)|^2 + 16Td\sigma^2\Big)e^{2dK^2T^2}.
\end{equation}
\end{theorem}
\begin{proof} The proof follows from Lemma \ref{lem:Lemma2}. Let us denote by $X_i(t), X_i^\epl(t)$ the $i$-th component of the solution, $i = 1, \dots, d$. Then
\begin{equation*}
\begin{aligned}
	\E\|X^\epl(t) - X(t)\|_2^2  &= \E \sum_{i=1}^d |X_i^\epl(t) - X_i(t)|^2 \\
		&= \sum_{i=1}^d \E|X_i^\epl(t) - X_i(t)|^2 \\
\end{aligned}
\end{equation*}
Then, applying Lemma \ref{lem:Lemma2} to each component of the solution, one obtains \eqref{eq:Proposition}.
\end{proof}

\begin{remark} \normalfont{\textbf{(Convergence considerations.)} The result of Proposition \ref{th:Prop} shows that in case one uses two different Wiener processes $W_1, W_2$ for the standard and the perturbed SDEs, the strong convergence is not granted. In fact, while the first term in estimation \eqref{eq:Proposition} disappears due to the uniform convergence of the perturbed transport field towards the non-perturbed one, the error due to the different Wiener processes is not disappearing with respect to $\epl$. However, in underground flow models the variance $\sigma$ of the Brownian diffusion is often small with respect to the transport field. Therefore, one could consider the solution $X^\epl(t)$ of \eqref{eq:AppSDEPer} to be practically converging to the solution $X(t)$ of \eqref{eq:AppSDE}.}
\end{remark}

\begin{remark} \normalfont{\textbf{(Interpolation results.)} Let us consider a domain $D \in \R^d$ where $f$ and $f^\epl$ are defined and $f^\epl$ to be a polynomial interpolation of $f$ on D. Then, the interpolation error of can be estimated if $f$ is regular enough. In particular, the following results are valid \cite[Chapter 8]{Quarteroni2007} for a grid of size $\epl$ 
\begin{enumerate}
	\item $f$ Lipschitz continuous of constant $K$, $f^\epl$ piecewise constant interpolation, $C_0$ a real constant
		\begin{equation*}
			\sup_{x\in D} \|f(x) - f^\epl(x)\| \leq C_0 K \epl.
		\end{equation*}
	\item $f$ of class $\mathcal{C}^1$, $f^\epl$ piecewise constant interpolation, $C_1$ a real constant
		\begin{equation*}
			\sup_{x\in D} \|f(x) - f^\epl(x)\| \leq C_1 \epl \sup_{x\in D} \|f'(x)\|.
		\end{equation*}
	\item $f$ of class $\mathcal{C}^2$, $f^\epl$ piecewise linear interpolation, $C_2$ a real constant
		\begin{equation*}
			\sup_{x\in D} \|f(x) - f^\epl(x)\| \leq C_2 \epl^2 \sup_{x\in D} \|f'(x)\|.
		\end{equation*}
\end{enumerate}
In all these cases, it is true that $f^\epl$ converges uniformly to $f$ with respect to $\epl$, therefore $f^\epl$ satisfies the assumptions of Proposition \ref{th:Prop}. }
\end{remark}



