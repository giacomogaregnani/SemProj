\section{Theoretical investigation}
In this section we analyse the impact on the analytic and numerical solution of an SDE given by a perturbation on the transport field. The theoretical investigation we present is needed in order to give a meaning to the results we find in the Darcy case. Since the velocity field is approximated by means of the Finite Elements method, we hope that the solution of an SDE that has the numerical approximation of the velocity field as transport term would converge to the solution of the SDE that uses the exact velocity field. Moreover, it can be costly to evaluate the Finite Element solution at each timestep of DEM or CEM, but an interpolation procedure can be exploited in order to obtain faster simulations. Hence, another source of perturbation is introduced, and a theoretical background is fundamental to strengthen the basis of our method.

\subsection{Analysis of perturbed SDEs}
Let us consider $(\Omega, \mathcal{A}, P)$ a complete probability space, $(W_1(t), t\geq 0), (W_2(t), t\geq 0)$ two $d$-dimensional standard Wiener processes not necessarily independent and a filtration $(\mathcal{F}(t), t \geq 0)$ such that $W_1(t), W_2(t)$ are $\mathcal{F}(t)$-measurable. Moreover, let us consider $\sigma \in \R,$ and a function $f\colon \R^d \rightarrow \R^d$ and the following SDE
\begin{equation}\label{eq:AppSDE}
\left \{
\begin{aligned}
	dX(t) &= f(X(t))dt + \sigma IdW_1(t), && 0 < t \leq T, \\
	X(0) &= X_0,
\end{aligned} \right .
\end{equation}
where $I$ is the identity matrix in $\R^{d\times d}$. Let us consider a perturbation of the transport field $f^\epl\colon \R^d \rightarrow \R^d$. Then, we consider the perturbed SDE 
\begin{equation}\label{eq:AppSDEPer}
\left \{
\begin{aligned}
	dX^\epl(t) &= f^\epl(X^\epl(t))dt + \sigma I dW_2(t), && 0 < t \leq T, \\
	X^\epl(0) &= X_0.
\end{aligned} \right .
\end{equation}
Finally, let us introduce the following notation. Given a function $F\colon \R^d \to \R$
\begin{equation*}
	\|F\|_{\infty} = \sup_{x\in \R^d} |F(x)|.
\end{equation*}
We can state the following preliminary result
\begin{lemma}\label{lem:Lemma1} With the notation above, let us consider $d = 1$, $W_1 = W_2 = W$ almost everywhere. If the following assumptions are verified for a constant $K > 0$
\begin{enumerate}
	\item $|f(x) - f(y)| \leq K|x - y|, \: \forall x,y \in \R$,
	\item $|f(x)| \leq K(1 + |x|), \: \forall x \in \R$,
\end{enumerate}
and if the solution $X^\epl(t)$ of \eqref{eq:AppSDEPer} exists, then $X^\epl(t)$ and $X(t)$ the solution of \eqref{eq:AppSDE} satisfy
\begin{equation*}
	\E \sup_{0 \leq t \leq T} |X^\epl(t) - X(t)|^2 \leq  2T^2 \|f - f^\epl\|_{\infty}^2 e^{2K^2T^2}.
\end{equation*}
\end{lemma}

\begin{proof}
For almost all $\omega \in \Omega$ 
\begin{equation*}
\begin{aligned}
	|X^\epl(t) - X(t)|^2  &= \Big|\int_0^t (f^\epl (X^\epl(s)) - f(X(s)))ds \Big|^2 \\
	&\leq T \int_0^t |f^\epl (X^\epl(s)) - f(X(s))|^2 ds  \\
	&\leq 2T \int_0^t |f^\epl (X^\epl(s)) - f(X^\epl(s))|^2 ds + 2T \int_0^t |f (X^\epl(s)) - f(X(s))|^2 ds \\
	&\leq 2T^2 \|f - f^\epl\|_{\infty}^2 + 2T^2K^2 \int_0^t |X^\epl(s) - X(s)|^2ds,
\end{aligned} 
\end{equation*}
where we applied Cauchy-Schwarz inequality, Young inequality, assumption 1 and the definition of $\|\cdot \|$ respectively. We then apply Gronwall's inequality, which gives 
\begin{equation}\label{eq:TheoryEpsilon}
	|X^\epl(t) - X(t)|^2 \leq  2T^2 \|f - f^\epl\|_{\infty}^2 e^{2K^2T^2}, \quad a.e.
\end{equation}
Since the right hand side of the inequality is independent of time and $\omega$, we can then take the supremum over time and the expectation at both sides, \textit{i.e.},
\begin{equation*}
	\E \sup_{0 \leq t \leq T} |X^\epl(t) - X(t)|^2 \leq  2T^2 \|f - f^\epl\|_{\infty}^2 e^{2K^2T^2},
\end{equation*}
which concludes the proof.
\end{proof}
\noindent We now consider the case of two independent Wiener processes.

\begin{lemma}\label{lem:Lemma2} With the notation above and the assumptions of Lemma \ref{lem:Lemma1}, let us consider $W_1$ independent of $W_2$ and $d = 1$. Then
\begin{equation*}	
	\E \sup_{0 \leq t \leq T} |X^\epl(t) - X(t)|^2 \leq 4T(T \|f - f^\epl\|_{\infty}^2 + 4\sigma^2) e^{2K^2T^2}.
\end{equation*}
\end{lemma}

\begin{proof} Let us compute the difference between $X^\epl(t)$ and $X(t)$. Applying Young's inequality, we get
\begin{equation*}
\begin{aligned}
	\E \sup_{0 \leq t \leq T} |X^\epl(t) - X(t)|^2 &\leq 2\E \sup_{0 \leq t \leq T}\Big|\int_0^t (f^\epl (X^\epl(s)) - f(X(s)))ds \Big|^2 \\
		&\quad + 2\sigma^2 \E \sup_{0 \leq t \leq T} \Big|\int_0^t dW_1(s) - \int_0^t dW_2(s)  \Big|^2.
\end{aligned}
\end{equation*}
Let us define $Z(t) := W_1(t) - W_2(t)$. The process $Z(t)$ is a Wiener process with variance $2t$, thus it is a martingale. Hence, we can apply Doob's maximal quadratic inequality (\textit{e.g.}, \cite[Page 11]{Protter2004}) to the second term, obtaining
\begin{equation*}
	\E \sup_{0 \leq t \leq T} \Big|\int_0^t dW_1(s) - \int_0^t dW_2(s)  \Big|^2 = \E \sup_{0 \leq t \leq T} | Z(t) |^2 \leq 4 \E |Z(T)|^2 = 8T.
\end{equation*}
For the first term, we can apply the same technique as in Lemma \ref{lem:Lemma1}. Therefore, we get the result
\begin{equation*}
	\E \sup_{0 \leq t \leq T} |X^\epl(t) - X(t)|^2 \leq (4T^2 \|f - f^\epl\|_{\infty}^2 + 16T\sigma^2) e^{2K^2T^2}.
\end{equation*}
\end{proof}
\noindent With those preliminary results, we can now consider the general case of a $d$-dimensional SDE.
\begin{theorem}\label{th:Prop} With the notation above and if there exists a real constant $K$ such that
\begin{enumerate}
	\item $\|f(x) - f(y)\| \leq K\|x - y\|, \: \forall x,y \in \R^d$,
	\item $\|f(x)\| \leq K(1 + \|x\|), \: \forall x \in \R^d$,
\end{enumerate}
then it is true for $X(t), X^\epl(t)$ the solutions of \eqref{eq:AppSDE} and \eqref{eq:AppSDEPer} that 
\begin{equation}\label{eq:Proposition}
	\E \sup_{0\leq t\leq T} \| X^\epl(t) - X(t) \|_2^2 \leq 4T\Big(T \sum_{i=1}^d \|f_i(x) - f_i^\epl(x)\|_\infty^2 + 4d\sigma^2\Big)e^{2dK^2T^2}.
\end{equation}
\end{theorem}
\begin{proof} The proof follows from Lemma \ref{lem:Lemma2}. Let us denote by $X_i(t), X_i^\epl(t)$ the $i$-th component of the solution, $i = 1, \dots, d$. Then
\begin{equation*}
\begin{aligned}
	\E\sup_{0\leq t\leq T} \|X^\epl(t) - X(t)\|^2  &= \E \sup_{0\leq t\leq T} \sum_{i=1}^d |X_i^\epl(t) - X_i(t)|^2 \\
		&\leq  \sum_{i=1}^d \E \sup_{0\leq t\leq T} |X_i^\epl(t) - X_i(t)|^2 \\
\end{aligned}
\end{equation*}
Then, applying Lemma \ref{lem:Lemma2} to each component of the solution, one obtains \eqref{eq:Proposition}.
\end{proof}

\begin{remark} \normalfont{\textbf{(convergence considerations.)} The result of Proposition \ref{th:Prop} shows that in case one uses two different Wiener processes $W_1, W_2$ for \eqref{eq:AppSDE} and \eqref{eq:AppSDEPer}, the strong convergence is not granted. In fact, while the first term in estimation \eqref{eq:Proposition} tends to zero due to the uniform convergence of the perturbed transport field towards the non-perturbed one, the error due to the different Wiener processes is independent of $\epl$. However, in underground flow models the variance $\sigma$ of the Brownian diffusion is often small with respect to the transport field. Therefore, one could consider the solution $X^\epl(t)$ of \eqref{eq:AppSDEPer} to be practically converging to the solution $X(t)$ of \eqref{eq:AppSDE} in case $f^\epl$ converges to $f$ with respect to $\epl$.}
\end{remark}

\begin{remark} \label{rem:Remark2} \normalfont{\textbf{(interpolation results.)} Let us consider an interval $D = [l, r] \subset \R$ where $f$ and $f^\epl$ are defined and $f^\epl$ to be a polynomial interpolation of $f$ on a grid $x_i = l + \epl i, i = 0, \dots, N, r = l + N\epl$. Then, the interpolation error of can be estimated if $f$ is regular enough. In particular, we can state the following results  
\begin{enumerate}
	\item If $f$ is Lipschitz continuous of constant $K$, and $f^\epl$ is its piecewise constant interpolation computed at the midpoint of each subinterval of the grid
		\begin{equation*}
			\sup_{x_i \leq x \leq x_{i+1}} |f(x) - f^\epl(x)| \leq K|x - r| \leq \frac{1}{2} K \epl, \: i = 0, \dots, N - 1.
		\end{equation*}
		Since $f$ is continuous
		\begin{equation*}
			\|f - f^\epl \|_{\infty} \leq \frac{1}{2} K \epl.
		\end{equation*}
	\item If $f$ is of class $\mathcal{C}^1$, $f^\epl$ is its piecewise constant interpolation and for a real constant $C_1$ \cite[Chapter 8]{Quarteroni2007}
		\begin{equation*}
			\|f - f^\epl\|_{\infty} \leq C_1 \epl \|f'(x)\|_{\infty}.
		\end{equation*}
	\item If $f$ is of class $\mathcal{C}^2$, $f^\epl$ is its piecewise linear interpolation and for a real constant $C_2$ \cite[Chapter 8]{Quarteroni2007}
		\begin{equation*}
			\|f - f^\epl\|_{\infty} \leq C_2 \epl^2 \|f''(x)\|_{\infty}.
		\end{equation*}
\end{enumerate}
In all these cases, $f^\epl$ converges uniformly to $f$ with respect to $\epl$, therefore the solution of \eqref{eq:AppSDEPer} converges to the solution of \eqref{eq:AppSDE} with respect to $\epl$. }
\end{remark}

\subsubsection{Analysis of numerical convergence}

We now consider the Euler-Maruyama method applied to the perturbed equation \eqref{eq:AppSDEPer}. We would like to find a balance between the error due to the numerical integration of the SDE with step size $h$ and the approximation of the transport field $f$ with $f^\epl$. In this way, one can choose wisely the two parameters $\epl, h$ in order to avoid extra computational time.

\noindent Let us consider $X(t), X^\epl(t)$ the solution of \eqref{eq:AppSDE} and \eqref{eq:AppSDEPer} respectively, and let us denote with $X_n, X_n^\epl$ the numerical solution obtained with Euler-Maruyama method applied to the two equations at time $t_n = hn, t_N = T$. It is known that if $f$ is Lipschitz continuous with constant $K$, and if $\sigma$ is a constant, then the strong error for $X_n$ approximating $X(t)$ is given by
\begin{equation}\label{eq:StrongEM}
	\sup_{n = 1, \dots, N} \E |X(nh) - X_n| \leq C h,
\end{equation}
for a constant $C$ independent of $h$. It is possible to obtain a similar estimate for $X_n^\epl$ estimating $X(t)$. 
\begin{theorem}\label{thm:StrongConv} Let us consider \eqref{eq:AppSDE}, \eqref{eq:AppSDEPer} with $d = 1$ and $W_1 = W_2 = W$ almost everywhere. Given $h > 0$ such that $hN = T$ for $N \in \N^*$, let us consider $X_n^\epl$ the numerical solution given by the Euler-Maruyama method applied to \eqref{eq:AppSDEPer}, \textit{i.e.},
\begin{equation*}
\left \{
\begin{aligned}
	X^\epl_{n+1} &= X^\epl_n + f^\epl(X^\epl_n)h + \sigma (W(t_{n+1}) - W(t_n)), && n = 0, \dots, N - 1, \\
	X^\epl_0 &= X_0.
\end{aligned} \right .
\end{equation*}
Then, if $f, f^\epl$ satisfy the assumptions 1. and 2. of Lemma \ref{lem:Lemma1}
\begin{equation}\label{eq:StrongConvEpl}
	\sup_{n = 1, \dots, N} \E |X(nh) - X^\epl_n| \leq C h + \sup_{x \in \R} |f^\epl(x) - f(x)| \frac{e^{KT} - 1}{K}, 
\end{equation}
with $C$ a real constant independent of $h$ and depending only on the final time $T$ and the Lipschitz constant $K$ of $f$.
\end{theorem}

\begin{proof} We consider $X_n$ the numerical approximation of $X(t)$ obtained using Euler-Maruyama with the same initial condition $X_0$. If we add and substract $X_n$ and apply the triangular inequality we obtain
\begin{equation*}
	\E |X_n^\epl - X(nh)| \leq \E |X_n^\epl - X_n| + \E |X_n - X(nh)|. \\
\end{equation*}
Since $f$ is regular enough and $\sigma$ is a constant, for the second term it is known that
\begin{equation}\label{eq:EMStrongProof}
	\sup_{n = 1, \dots, N} \E |X_n - X(nh)| \leq C h.
\end{equation}
We can then make a recursive analysis of the first term. For almost all $\omega$
\begin{equation*}
\begin{aligned}
	|X^\epl_n - X_n| &\leq |X_{n-1}^\epl - X_{n-1}| + h|f^\epl(X^\epl_{n-1}) - f(X_{n-1})|  \\
	&\leq |X_{n-1}^\epl - X_{n-1}| + h|f^\epl(X^\epl_{n-1}) - f(X^\epl_{n-1})| + h|f(X^\epl_{n-1}) - f(X_{n-1})| \\
	&\leq h \sup_{x \in \R} |f^\epl(x) - f(x)| + (1 + hK) |X_{n-1}^\epl - X_{n-1}| \\
	&\leq h \sup_{x \in \R} |f^\epl(x) - f(x)| \\
	&\quad + (1 + hK) (h \sup_{x \in \R} |f^\epl(x) - f(x)| + (1 + hK)|X_{n-2}^\epl - X_{n-2}|) \\
	&\leqtext{(\cdots)} h \sup_{x \in \R} |f^\epl(x) - f(x)| \sum_{i = 0}^{n-1} (1 + hK)^i + (1 + hK)^n |X_0^\epl - X_0|
\end{aligned}
\end{equation*}
Since $X_0^\epl = X_0$ and using the geometric sum
\begin{equation*}
\begin{aligned}
	|X^\epl_n - X_n| &\leq h \sup_{x \in \R} |f^\epl(x) - f(x)| \frac{(1 + hK)^n - 1}{hK} \\
	&\leq \sup_{x \in \R} |f^\epl(x) - f(x)| \frac{(1 + hK)^N - 1}{K} \\
	&\leq \sup_{x \in \R} |f^\epl(x) - f(x)| \frac{e^{KT} - 1}{K},
\end{aligned}
\end{equation*}
where the last inequality is valid since $N = T/h$ and $K, T, h$ are all positive real numbers. Since the bound we found is independent of $\omega$ and $n$, we can take the expectation and the supremum, obtaining
\begin{equation*}
	\sup_{n = 0, \dots, N} \E |X^\epl_n - X_n| \leq \sup_{x \in \R} |f^\epl(x) - f(x)| \frac{e^{KT} - 1}{K}.
\end{equation*}
This result combined with \eqref{eq:EMStrongProof} concludes the proof.
\end{proof}
\begin{remark} \normalfont{If $f^\epl$ is the interpolation of $f$ on a regular grid of size $\epl$, the result of Proposition \ref{thm:StrongConv} allows to balance the interpolation error and the error due to numerical integration. In fact, we reported above some results on interpolation of $f$ with piecewise polynomials $f^\epl$, which we can plug in \eqref{eq:StrongConvEpl} as follows 
\begin{enumerate}
	\item if $f$ is Lipschitz continuous of constant $K$ or of class $\mathcal{C}^1$ and $f^\epl$ is piecewise constant, then 
	\begin{equation*}
		\sup_{n = 0, \dots, N} \E |X^\epl_n - X_n| = O(h) + O(\epl).
	\end{equation*}
	\item if $f$ is of class $\mathcal{C}^2$ , and $f^\epl$ is piecewise linear, then 
	\begin{equation*}
		\sup_{n = 0, \dots, N} \E |X^\epl_n - X_n| = O(h) + O(\epl^2).
	\end{equation*}
\end{enumerate}
Therefore, in the first case the step size $h$ should be of the same order of magnitude as $\epl$, while in the second case it should scale as $\epl^2$.}\end{remark}

\subsection{Numerical confirmation of the theory}

We perform a numerical experiment in order to verify the theoretical bounds presented above. Let us consider the domain $D = [-1, 1]^2$ and the deterministic transport field given by
\begin{equation*}
	f(x, y) = \frac{1}{2} \begin{pmatrix} x^2 + y^2, & |x - y| \end{pmatrix}^T.
\end{equation*}
In the domain, the transport field is Lipschitz continuous. Our aim is verifying whether the numerical solution computed using a piecewise constant interpolation of $f$ leads to convergence with respect to the characteristic size of the grid used for interpolation. In the following, $f^\epl$ denotes the piecewise constant interpolation over $D$ of $f$ on a structured grid of equal size $\epl$ in both directions. Since the theoretical results interest the value of the solution itself and not the mean exit time from a domain, we start by considering the value itself. Then, we verify if the results are valid for the exit time as well.

\vspace{2mm}
\noindent \textbf{Numerical solution.} We consider the boundary conditions to be reflecting on all the boundary of $D$. We fix the parameters to be $T = 1, \sigma = 1$ and perform a Montecarlo simulation over $M = 10000$ simulations of Euler-Maruyama. Let us remark that since the boundary is completely reflecting, there is no distinction between CEM and DEM. Mantaining the notation of Proposition \ref{thm:StrongConv}, we compute the weak error of $X_n^\epl$ with respect to $X_n$, which should be of order $O(\epl)$. The reference solution $X_n$ is computed over $N = 2^{10}$ timesteps over the time span. Then, we vary $\epl$ in the range $\epl_i = 2^{-i}, i = 1,\dots, 8$ and compute the numerical solution $X_n^\epl$ either fixing the number of timesteps $N = 2^9$ or varying it so that $h_i = \epl_i$, \textit{i.e.}, the time and space discretizations have the same order of magnitude. In the first case, we wish to overkill the error due to numerical integration, while in the second case we wish to mantain an acceptable computational cost. Moreover, the bound presented in Remark \ref{rmk:InterpNum} implies that theoretically the error should be independent of $h$, so we would expect similar results for both the approaches. Results (Figure \ref{fig:TheoryX}) confirm the theoretical results, with a clear rate of convergence equal to one, and the errors which are similar for the two approaches.

\vspace{2mm}
\noindent \textbf{Mean exit time.} We perform an experiment with the same values for all parameters as above, but we consider mixed killing and reflecting boundary conditions and we estimate the exit time $\tau$. We wish that the theoretical results obtained for the solution apply practically to the exit time itself. We use the same strategies as above, either fixing a small step size $h$ for any value of $\epl$, or mantaining $h$ equal to $\epl$. Results (Figure \ref{fig:TheoryTau}) show that the error is $O(\epl)$ in both cases, with a smaller constant in case a small value of $h$ is chosen. On the other side, we notice that fixing, \textit{e.g.}, the error to $0.01$, the computational time is in this case approximately sixteen times smaller in case $h$ and $\epl$ are balanced.

\begin{figure}[t]
    \centering
    \begin{subfigure}{0.49\linewidth}
        \centering
        \resizebox{1\linewidth}{!}{\input{Darcy/Pictures/TheoryX.tikz} }   
        \caption{Error on the solution.}
        \label{fig:TheoryX}
    \end{subfigure}
    \begin{subfigure}{0.49\linewidth}
        \centering
        \resizebox{1\linewidth}{!}{\input{Darcy/Pictures/TheoryTau.tikz} }  
        \caption{Error on the exit time.}
        \label{fig:TheoryTau}
    \end{subfigure}    
    \caption{Convergence of the numerical solution with respect to the interpolation characteristic size $\epl$.}
    \label{fig:Theory}
\end{figure}



