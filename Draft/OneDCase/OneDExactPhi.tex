\subsubsection{Numerical approximation of $\Phi$ with the PDE approach}

Let us consider $D$ as the interval $\left[ l,r \right]$, the boundary condition in $l$ to be fixed to killing and in $r$ to be either killing or reflecting. In this case and for $f,g$ independent of $t$, \eqref{eq:PDEPhi} can be written as the following initial value PDE 
\begin{equation}\label{eq:PDEPhiOneD}
\left \{
\begin{aligned}
	-\frac{\partial}{\partial t} \Phi + f\frac{\partial}{\partial x} \Phi + \frac{1}{2}g^2 \frac{\partial^2}{\partial x^2} \Phi &= 0, && l < x < r \\
	\Phi(l) &= 1, \\
	\Phi(r) &= 1, && \text{if for $x = r$ the boundary is \textit{killing}} \\
	\frac{\partial}{\partial x}\Phi(r) &= 0, && \text{if for $x = r$ the boundary is \textit{reflecting}} \\
\end{aligned} \right .
\end{equation}
This equation can be solved, \textit{e.g.}, using finite differences. If the implicit Euler scheme is employed for integration on a equispaced grid in space defined by the interval $\Delta_x$ and in time by the step size $\Delta_t$, denoting by $\Phi_j^k$ the solution at the $j$-th node and the $k$-th time point and by $f_j = f(j\Delta_x), g_j = g(j\Delta_x)$, the solution is found solving iteratively the following linear system
\begin{equation}\label{eq:FDPhi}
	- \frac{1}{2}\Big(-f_j \frac{\Delta_t}{\Delta_x} + g_j^2 \frac{\Delta_t}{\Delta_x^2}\Big)\Phi_{j-1}^{k+1} + \Big(1 + g_j^2 \frac{\Delta_t}{\Delta_x^2}\Big)\Phi_j^{k+1} - \frac{1}{2}\Big(f_j \frac{\Delta_t}{\Delta_x} + g_j^2 \frac{\Delta_t}{\Delta_x^2}\Big)\Phi_{j+1}^{k+1}  = \Phi_j^k.
\end{equation}
The solution of this equation does not imply an high computational time, therefore one can choose $\Delta_x$ and $\Delta_t$ to be small. Proceeding in this way, the value of $\Phi$ obtained with finite differences could be used as a reference value for estimating the error of DEM and CEM.
