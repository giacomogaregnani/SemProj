\subsubsection{Numerical approximation of $\Phi$ with the PDE approach}

Let us consider $D$ as the interval $\left[ l,r \right]$, the boundary condition in $l$ to be fixed to killing and in $r$ to be either killing or reflecting. In this case and for $f$ independent of $t$ and $g = \sigma \in \mathbb{R}$ \eqref{eq:PDEPhi} can be written as the following initial value PDE 
\begin{equation}\label{eq:PDEPhiOneD}
\left \{
\begin{aligned}
	-\frac{\partial}{\partial t} \Phi(t,x) + f\frac{\partial}{\partial x} \Phi(t,x) + \frac{1}{2}\sigma^2 \frac{\partial^2}{\partial x^2} \Phi(t,x) &= 0, && l < x < r \\
	\Phi(t,l) &= 1, \\
	\Phi(t,r) &= 1, && \text{if for $x = r$ the boundary is \textit{killing}} \\
	\frac{\partial}{\partial x}\Phi(t,r) &= 0, && \text{if for $x = r$ the boundary is \textit{reflecting}} \\
	\Phi(0,x) &= 0.
\end{aligned} \right .
\end{equation}
This equation can be solved, \textit{e.g.}, using finite differences. We employ the theta method for solving \eqref{eq:PDEPhiOneD}. Let us consider the case in which $r$ is a killing boundary, $i.e.$, the PDE is endowed with Dirichlet boundary conditions. Given a step size $\Delta_t$ for time integration and an uniform grid $x_i = l + i\Delta_x, i=0,\dots,N+1, x_{N+1} = r$, at each timestep $k$ one has to find the solution of the linear system
\begin{equation}\label{eq:ThetaMethod}
	(I - \Delta_t\theta A) u^{k+1} = (I + \Delta_t(1-\theta) A)u^k + hF, \: 0 \leq \theta \leq 1,
\end{equation}
where $I$ is the identity matrix of $\mathbb{R}^{N\times N}$. The matrix $A$ of $\mathbb{R}^{N\times N}$ and the vector $F$ of $\mathbb{R}^N$ define the space discretization and the boundary conditions and are defined by
\begin{equation}\label{eq:ThetaMethodAandF}
	A = \frac{1}{2\Delta_x}\begin{pmatrix} 	\alpha_1 & \beta_1  &  	      &\\
						\gamma_1 & \alpha_2 & \beta_2 &\\
							 & \ddots   & \ddots  & \ddots \end{pmatrix}, \quad F = \frac{1}{2\Delta_x}\begin{pmatrix} F_1 \cdots F_N \end{pmatrix}^T
\end{equation}
and the coefficients are given by
\begin{equation}
\begin{split}
	\alpha_i &= -\frac{2\sigma^2}{\Delta_x}, \: i = 1, \dots, N, \\
	\beta_i  &= \frac{\sigma^2}{\Delta_x} + f(x_{i}), \: i = 1, \dots, N-1, \\
	\gamma_i &= \frac{\sigma^2}{\Delta_x} - f(x_{i+1}), \: i = 1, \dots, N-1, \\
	F_1      &= \frac{\sigma^2}{\Delta_x} - f(x_1), \\
	F_N      &= \frac{\sigma^2}{\Delta_x} - f(x_{N-1}).
\end{split}	
\end{equation}
The case of reflecting boundary condtion in $x = r$ is similar and affects only the computation of the matrix $A$ and the vector $F$. In particular, we introduce a \textit{ghost node} at position $x = r + \Delta_x$, compute the derivative using a centrate approximation and impose that it is equal to 0, which leads to the condition that the value in the node in $x = r$ is equal to the value in $x = r - \Delta_x$. Since the matrix defining the system \eqref{eq:ThetaMethod} is tridiagonal, one can choose $\Delta_t, \Delta_x$ to be small and obtain a precise solution of \eqref{eq:PDEPhiOneD} in a reasonable computational time. In the following, we will compare the values given by Montecarlo simulations using DEM and CEM with the solution of the theta method with $\theta = 0.5$.

 
