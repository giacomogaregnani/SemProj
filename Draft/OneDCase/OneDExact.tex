\subsubsection{Analytical expression of the mean exit time}
In this simple frame, it is possible to deduce an analytical solution $\bar\tau$ of \eqref{eq:PDETau}. Let us consider the boundary condition at $x=l$ fixed as \textit{killing} and vary the boundary condition at $x=r$. Since the scope is deducing the exit time of a particle from $D$, this assumption is plausible. In this frame, it is possible to rewrite \eqref{eq:PDETau} as
\begin{equation}\label{eq:ODETau}
\begin{cases}
	f(x)\bar\tau'(x) + \frac{1}{2} g^2(x) \bar\tau''(x) = -1, & l < x < r, \\
	\bar\tau(l) = 0, \\
	\bar\tau(r) = 0, & \text{if for $x = r$ the boundary is \textit{killing}}, \\
	\bar\tau'(r) = 0, & \text{if for $x = r$ the boundary is \textit{reflecting}}. 
\end{cases}
\end{equation}
It is possible to show \cite{Krumscheid2015,Pavliotis2014} that $\bar\tau$ is in the one-dimensional case given by
\begin{equation}\label{eq:AnalyticTau}
	\bar\tau(x) = -2 \int_l^x \exp(-\psi(z)) \int_l^z \frac{\exp(\psi(y))}{g^2(y)}dy + c_1 \int_l^x \exp(-\psi(y))dy + c_2,
\end{equation}
where the function $\psi$ is defined as
\begin{equation}\label{eq:psi}
	\psi(x) = \int_l^x \frac{2f(y)}{g^2(y)}dy,
\end{equation}
and the constants $c_1,c_2 \in \mathbb{R}$ depend on the boundary conditions as follows
\begin{align}\label{eq:Constants}
\begin{split}
	c_1 &= 2\frac{\int_l^r \exp(-\psi(z)) \int_l^z \frac{\exp(\psi(y))}{g^2(y)}dy}{\int_l^r \exp(-\psi(y))dy}, \text{  if for $x = r$ the boundary is \textit{killing}}, \\
	c_1 &= 2\int_l^r \frac{\exp(-\psi(y))}{g(y)^2}dy, \text{  if for $x = r$ the boundary is \textit{reflecting}}, \\
	c_2 &= 0.
\end{split}
\end{align}
Let us remark that in case $f = -V'$ for some smooth function $V$ and $g = \sigma \in \mathbb{R}$, the expression of $\psi$ semplifies to
\begin{equation}\label{eq:psiSemplified}
	\psi(x) = 2\frac{V(l)-V(x)}{\sigma^2}.
\end{equation}
The value for the expected exit time given by \eqref{eq:AnalyticTau} will be used as a reference for verifying the order of convergence of the numerical methods.
