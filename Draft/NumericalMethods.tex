\subsection{Numerical Methods}

\subsubsection{Discrete Euler-Maruyama}
Given $N \in \mathbb{N}$ let us define a partition of $[0,T]$ as $P_h = \{t_i\}_{i=0}^{N}, t_i = ih, h = T/N$. The Discrete Euler-Maruyama method (DEM) for problem \eqref{eq:GeneralModel} is defined as follows
\begin{equation}\label{eq:DEM}
	\left \{
	\begin{aligned}
		X_h^d(t_{i+1}) &= f(X(t_i))h + g(X(t_i))(W(t_{i+1}) - W(t_{i})),  \\
		X_h^d(0) &= X_0.
	\end{aligned} \right .
\end{equation} 
The exit time $\tau$ is approximated with the quantity $\tau_h^d$ defined as 
\begin{equation}\label{eq:TauDEM}
	\tau_h^d = \min\{\min \{t_i \colon X_h^d(t_i) \notin D\},T\}.
\end{equation}
We approximate analogously $\phi$ as
\begin{equation}\label{eq:PhiDEM}
	 \phi_h^d = \mathbbm{1}_{\{\tau_h^d < T\}}F(X_h^d(T)).
\end{equation}
Let us state two results concerning the weak error of this method.
\begin{theorem}\label{ThDEMTau} Under appropriate assumptions of smoothness of $f,g,D,\partial D,F$,
\begin{equation}\label{eq:ConvDEMTau}
	|\mathbb{E}(\tau_h^d) - \mathbb{E}(\tau)| = O(\sqrt{h}).
\end{equation}
\end{theorem}
\begin{theorem}\label{ThDEMPhi} Under approprate assumptions of smoothness of $f,g,D,\partial D,F$,
\begin{equation}\label{eq:ConvDEMPhi}
	|\mathbb{E}(\phi_h^d) - \mathbb{E}(\phi)| = O(\sqrt{h}).
\end{equation}	
\end{theorem}
An discussion of result \ref{ThDEMTau} can be found in \cite{Higham2013}, its proof in \cite{Gobet2010}. A proof of \ref{ThDEMPhi} can be found in \cite{Gobet2000}.

\subsubsection{Continuous Euler-Maruyama. }
Let us consider the partition $P_h$ of $[0,T]$ as above. The Continuous Euler-Maruyama (CEM) method is defined as
\begin{equation}\label{eq:CEM}
	\left \{
	\begin{aligned}
		X_h^c(t) &= f(X(t_i))(t-t_i) + g(X(t_i))(W(t) - W(t_{i})),  && t_i < t \leq t_{i+1},\\
		X_h^c(0) &= X_0.
	\end{aligned} \right .
\end{equation} 
Let us remark that in case the particle does not exit the domain, $X_h^c(t_i) = X_h^d(t_i)$ for all $t_i \in P_h$. It is possible to compute the probability that a particle has exited the domain at a time $t$ between two consecutive timesteps $t_i,t_{i+1}$ when $D$ is an half-space with the following formula \cite{Gobet2001}
\begin{equation}\label{eq:CEMProb}
	\Pr (\exists t \in [ t_i,t_{i+1} ] \quad X_h^d(t) \notin D | X_h^d(t_i) = x_i, X_h^d(t_{i+1}) = x_{i+1}) = p(x_i,x_{i+1},h),
\end{equation}
with $p(x_i,x_{i+1},h)$ given by
\begin{equation}\label{eq:CEMProbHalfSpace}
	p(x_i,x_{i+1},h) = \exp\Big(-2\frac{[n\cdot(x_i - z_i)][n\cdot(x_{i+1} - z_i)]}{hn\cdot (gg^T(x_i)n)}\Big),
\end{equation}
where $z_i$ is the projection of $x_i$ on $\partial D$ and $n$ is the normal to $\partial D$ in $z_i$. At each timestep $t_{i+1}$ we compute the probability $p(x_i,x_{i+1},h)$, and then simulate a variable $U$ distributed uniformly in the interval $\left[0,1\right]$, thus obtaining a realization $u$. Hence, we counclude that the particle has left the domain for a time $t$ in $(t_i,t_{i+1})$ if $u$ is smaller than $p$. Therefore, we approximate the exit time as
\begin{equation}\label{eq:TauCEM}
	\tau_h^c = \min \{T,\min\{t_i = hi \colon X_h(t_i) \notin D\}, \min\{t_i = hi \colon u < p(x_{i-1},x_i,h) \}\},
\end{equation}
In the same way as in DEM, we can approximate $\phi$ as
\begin{equation}\label{eq:PhiCEM}
	\phi_h^c = \mathbbm{1}_{\{\tau_h^c < T\}}F(X_h^c(T)).
\end{equation}
We show the pseudocode for the implementation of CEM in Algorithm \ref{alg:algoCEM}.

\begin{algorithm}[t]
\caption{Continuous Euler-Maruyama}
\For{$t_i \in P_h$ }{
	$X(t_{i+1}) = f(X(t_i))h + g(X(t_i))(W(t_{i+1})-W(t_i))$ \;
  	\eIf{$X(t_{i+1}) \notin D$}{
    		$\tau_h^c = t_{i+1}$ \;
		$\phi_h^c = F(X_h^c(t_{i+1}))$ \;
		\Return \;
   	}{
   	compute $p = p(x_i,x_{i+1},h)$ \;
	simulate $u \sim$ Unif$(0,1)$ \;
	\If{$u < p$}{
		$\tau_h^c = t_{i+1}$ \;
		$\phi_h^c = F(X_h^c(t_{i+1}))$ \;
		\Return \;
		}
  	}
 }
\label{alg:algoCEM}
\end{algorithm}

\noindent The weak error of this method has been studied exhaustively in previous work.
\begin{theorem}\label{ThCEMPhi} Under appropriate smoothness assumptions, 
\begin{equation}\label{eq:ConvCEMPhi}
	|\mathbb{E}(\phi_h^c) - \mathbb{E}(\phi)| = O(h).
\end{equation}
\end{theorem}
A proof of result \ref{ThCEMPhi} can be found in \cite{Gobet2001}.

\subsubsection{Reflecting boundaries}
The reflecting boundaries are treated in the same way for both DEM and CEM. Let us denote by $\Gamma_k$ and $\Gamma_r$ the killing and reflecting subsets of $\partial D$, \textit{i.e.}
\begin{equation}\label{eq:Boundaries}
	\Gamma_r \cup \Gamma_k = \partial D, \quad \Gamma_r \cap \Gamma_k = \emptyset
\end{equation} 
In case the particle approaches $\Gamma_k$ the exit is treated as above. If for a timestep of $t_i \in P_h$, $X(t_i)$ is not in $D$ and has crossed $\Gamma_r$ at a time $t_{i-1} < t < t_i$, we update the solution to be the normal reflection inside $D$ of $X(t_i)$.

\subsubsection{Adaptivity}
We apply to a test case the adaptivity procedure explained in section \ref{sec:Adapt}. In particular, we fix $f = 0, g = \sigma = 1, T = 3$ and $X_0 = (0,0)^T$ in the square domain $D = \left[-1,1\right]^2$ with pure killing boundary conditions. We vary $l$ in \eqref{eq:Adaptivity} in the range $l = 0, \dots, 7$ and we consider three methods
\begin{enumerate}
	\item Adaptive method with $h_0 = T, h_{int} = 2^{-l}h_0, h_{bound} = 2^{-2l}h_0$,
	\item DEM with $h = h_{bound}$ for each timestep,
	\item CEM with $h = h_{int}$ for each timestep,
\end{enumerate}
Since $h_{bound} \sim h_ {int}^2$, we expect the three methods to have an error of order $O(h_{int})$ when estimating the mean exit time. In Figure \ref{fig:AdaptErr} it is possible to remark that the error is in fact of order $O(h_{int})$ for the three methods. Moreover, the methods have approximately the same error, \textit{i.e.}, the constant multiplying $h_{int}$ is the same for the three methods. In order to choose which method performs better from the point of view of computational cost, we compute the mean number of timesteps that they perform in order to estimate the exit time. From Figure \ref{fig:AdaptCost} it is clear that CEM with $h_{int}$ for every timestep performs better than the other two methods, with the adaptive procedure applied to DEM which implies lower computational time than DEM with constant timestep equal to $h_{bound}$. We can conclude that the approach based on Brownian bridge proposed in \cite{Gobet2001} has better performances when estimating the mean exit time from a domain.

\begin{figure}[t]
    \centering
    \begin{subfigure}{0.49\linewidth}
        \centering
        \resizebox{1\linewidth}{!}{% This file was created by matlab2tikz.
%
%The latest updates can be retrieved from
%  http://www.mathworks.com/matlabcentral/fileexchange/22022-matlab2tikz-matlab2tikz
%where you can also make suggestions and rate matlab2tikz.
%
\definecolor{mycolor1}{rgb}{0.00000,0.44700,0.74100}%
%
\begin{tikzpicture}

\begin{axis}[%
width=4.521in,
height=3.507in,
at={(0.758in,0.54in)},
scale only axis,
xmode=log,
xmin=0.01,
xmax=10,
xminorticks=true,
xlabel={$\text{h}_{\text{int}}$},
xmajorgrids,
xminorgrids,
ymode=log,
ymin=0.001,
ymax=10,
yminorticks=true,
ylabel={error},
ymajorgrids,
yminorgrids,
axis background/.style={fill=white},
legend style={at={(0.03,0.97)},anchor=north west,legend cell align=left,align=left,draw=white!15!black}
]
\addplot [color=red,solid,mark=o,mark options={solid}]
  table[row sep=crcr]{%
3	2.41079469931969\\
1.5	0.860169699319686\\
0.75	0.378032199319685\\
0.375	0.166808761819686\\
0.1875	0.0848777132218849\\
0.09375	0.0376179707148839\\
0.046875	0.026672599811998\\
0.0234375	0.0141627398021634\\
};
\addlegendentry{adaptive};

\addplot [color=blue,solid,mark=asterisk,mark options={solid}]
  table[row sep=crcr]{%
3	2.41079469931969\\
1.5	0.861894699319686\\
0.75	0.364382199319686\\
0.375	0.157658761819686\\
0.1875	0.0787564180696855\\
0.09375	0.0362689180696855\\
0.046875	0.0191677461946855\\
0.0234375	0.00641772788413864\\
};
\addlegendentry{$\text{DEM, h}_{\text{bound}}$};

\addplot [color=mycolor1,solid,mark=triangle,mark options={solid,rotate=90}]
  table[row sep=crcr]{%
3	2.41079469931969\\
1.5	1.09844469931969\\
0.75	0.469194699319685\\
0.375	0.209619699319685\\
0.1875	0.102482199319685\\
0.09375	0.0472540743196855\\
0.046875	0.0251712618196854\\
0.0234375	0.00993923056968549\\
};
\addlegendentry{$\text{CEM, h}_{\text{int}}$};

\addplot [color=black,solid]
  table[row sep=crcr]{%
3	1.33447009455748\\
1.5	0.667235047278742\\
0.75	0.333617523639371\\
0.375	0.166808761819686\\
0.1875	0.0834043809098428\\
0.09375	0.0417021904549214\\
0.046875	0.0208510952274607\\
0.0234375	0.0104255476137303\\
};
\addlegendentry{$\text{h}_{\text{int}}$};

\end{axis}
\end{tikzpicture}% }  
        \caption{Convergence of the methods.}
        \label{fig:AdaptErr}
    \end{subfigure}
    \begin{subfigure}{0.49\linewidth}
        \centering
        \resizebox{1\linewidth}{!}{% This file was created by matlab2tikz.
%
%The latest updates can be retrieved from
%  http://www.mathworks.com/matlabcentral/fileexchange/22022-matlab2tikz-matlab2tikz
%where you can also make suggestions and rate matlab2tikz.
%
\definecolor{mycolor1}{rgb}{0.00000,0.44700,0.74100}%
%
\begin{tikzpicture}

\begin{axis}[%
width=4.521in,
height=3.507in,
at={(0.758in,0.54in)},
scale only axis,
xmode=log,
xmin=0.01,
xmax=10,
xminorticks=true,
xlabel={$h_{int}$},
xlabel style = {font=\Large},
xmajorgrids,
xminorgrids,
ymode=log,
ymin=1,
ymax=10000,
yminorticks=true,
ylabel={Mean number of timesteps},
ylabel style = {font=\Large},
ymajorgrids,
yminorgrids,
axis background/.style={fill=white},
legend style={legend cell align=left,align=left,draw=white!15!black,font=\Large}
]
\addplot [color=red,solid,mark=o,mark options={solid}]
  table[row sep=crcr]{%
3	1\\
1.5	1.9325\\
0.75	5.1586\\
0.375	16.1283\\
0.1875	54.5378\\
0.09375	133.727\\
0.046875	262.3522\\
0.0234375	437.5306\\
};
\addlegendentry{adaptive};

\addplot [color=blue,solid,mark=asterisk,mark options={solid}]
  table[row sep=crcr]{%
3	1\\
1.5	1.9348\\
0.75	5.0858\\
0.375	15.9331\\
0.1875	56.9994\\
0.09375	213.4952\\
0.046875	830.632\\
0.0234375	3252.8959\\
};
\addlegendentry{$\text{DEM },h_{bound}$};

\addplot [color=mycolor1,solid,mark=triangle,mark options={solid,rotate=90}]
  table[row sep=crcr]{%
3	0\\
1.5	1\\
0.75	1.3953\\
0.375	2.1258\\
0.1875	3.6868\\
0.09375	6.7879\\
0.046875	13.1052\\
0.0234375	25.5623\\
};
\addlegendentry{$\text{CEM },h_{int}$};

\end{axis}
\end{tikzpicture}%
 }  
        \caption{Computational cost.}
        \label{fig:AdaptCost}
    \end{subfigure}    
    \caption{Estimation of $\tau$ in the square domain with mixed boundary conditions. Comparison between DEM with fixed or adaptive step size.}
    \label{fig:AdaptResults}
\end{figure}





