\subsubsection{An adaptive procedure}\label{sec:Adapt}

Let us consider the function $g$ in \eqref{eq:GeneralModel} to be fixed to a real constant $\sigma$. The numerical methods we presented can be completed with an adaptive procedure for the step size $h$. In this frame, adaptivity is performed with respect to the position of the numerical solution $X_h(t_i)$, \textit{i.e.}, the step size is reduced if the solution is near to the boundary of the domain $D$. If we denote by $d_i$ the distance between $X_h(t_i)$ and its normal projection on $\partial D$, by $l \in \N$ a fixed parameter and by $h_0$ a maximum bound for $h$, the step size is chosen using the following formula
\begin{equation}\label{eq:Adaptivity}
	h = \max\Big\{ 2^{-2l}h_0, \min\Big\{ 2^{-l}h_0, \Big(\frac{d}{(l + 3)\sigma}\Big)^2\Big\}\Big\}.
\end{equation}
This formula is equivalent to dividing the domain $D$ in three zones, in particular
\begin{itemize}
	\item an interior zone where $h = h_{int} = 2^{-l}h_0$,
	\item a boundary zone where $h = h_{bound} = 2^{-2l}h_0$,
	\item an intermediate zone where $h = \frac{d}{(l + 3)\sigma}^2$.
\end{itemize}
Since the loss of $0.5$ in the weak order of convergence of Euler-Maruyama is due to the non-detected exits from the domain $D$, refining the step size only near to the boundary could lead to an error equal to the one obtained using $h_{bound}$ on the whole domain.
