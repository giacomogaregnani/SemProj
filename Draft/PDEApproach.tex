\subsection{A PDE approach}\label{sec:PDEs}
It is possible to express the exit time and the probability of exit from a domain in terms of the solution of partial differential equations (PDE's).
Let us denote by $\Gamma_k,\Gamma_r$ the killing and reflecting subsets of $\partial D$. We consider then the expectation of the exit time from the domain $D$ for a trajectory that at $t=0$ is at position $x$, \textit{i.e.},
\begin{equation}\label{eq:ExpTau}
	\bar\tau = \mathbb{E}(\tau | X(0) = x).
\end{equation}
Let us define the operator $\mathcal L$ induced by \eqref{eq:GeneralModel} acts on a function $u\colon \mathbb{R}^d \rightarrow \mathbb{R}$  as follows
\begin{equation}\label{eq:LOperator}
	\mathcal Lu = f \cdot \nabla u + \frac{1}{2} gg^T : \nabla \nabla u,
\end{equation}
where the $:$ operator between two matrices $A,B$ in $\mathbb{R}^{d\times d}$ is defined as follows
\begin{equation}\label{eq:twoPoints}
	A : B = \sum_{i,j = 1}^d \{A\}_{ij}\{B\}_{ij} = \text{tr}(A^TB).
\end{equation}
It is possible to show \cite{Krumscheid2015,Pavliotis2014} that $\bar\tau$ is the solution of the following partial differential equation 
\begin{equation}\label{eq:PDETau}
\begin{cases}
	\mathcal L \bar \tau = -1, & \text{in } D, \\
	\bar\tau = 0 & \text{on } \Gamma_k, \\
	\nabla \bar\tau \cdot n = 0 & \text{on } \Gamma_r.
\end{cases}
\end{equation}
We now consider the probability of exit from $D$ for a solution $X(t)$ that is equal to $x$ for $t = s < T$. Let us introduce the following notation
\begin{equation}\label{eq:ExitProbNotation}
	\
\end{equation}
It is possible to approximate the solution $\bar\tau$ with a function $\bar\tau_h$ obtained with a classical method for solving PDE's, such as finite differences or the Finite Elements Method.

