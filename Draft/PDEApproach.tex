\subsection{A PDE approach}\label{sec:PDEs}
It is possible to express the exit time and the probability of exit from a domain in terms of the solution of partial differential equations (PDE's).
Let us denote by $\Gamma_k,\Gamma_r$ the killing and reflecting subsets of $\partial D$. We consider then the expectation of the exit time from the domain $D$ for a trajectory that at $t=0$ is at position $x$, \textit{i.e.},
\begin{equation}\label{eq:ExpTau}
	\bar\tau = \mathbb{E}(\tau | X(0) = x).
\end{equation}
Let us define the operator $\mathcal L$ induced by \eqref{eq:GeneralModel} acts on a function $u\colon \mathbb{R}^d \rightarrow \mathbb{R}$  as follows
\begin{equation}\label{eq:LOperator}
	\mathcal Lu = f \cdot \nabla u + \frac{1}{2} gg^T : \nabla \nabla u,
\end{equation}
where the $:$ operator between two matrices $A,B$ in $\mathbb{R}^{d\times d}$ is defined as follows
\begin{equation}\label{eq:twoPoints}
	A : B = \sum_{i,j = 1}^d \{A\}_{ij}\{B\}_{ij} = \text{tr}(A^TB).
\end{equation}
It is possible to show \cite{Krumscheid2015,Pavliotis2014} that $\bar\tau$ is the solution of the following partial differential equation 

\begin{theorem} Let $\mathcal L$ be the differential operator defined as \eqref{eq:LOperator}. Then, the mean exit time $\bar \tau$ is the solution of the following boundary value problem
\begin{equation}\label{eq:PDETau}
\begin{cases}
	\mathcal L \bar \tau = -1, & \text{in } D, \\
	\bar\tau = 0 & \text{on } \Gamma_k, \\
	\nabla \bar\tau \cdot n = 0 & \text{on } \Gamma_r.
\end{cases}
\end{equation}
\end{theorem}
Further analytical treatement of the mean exit time can be found in \cite{Krumscheid2015,Pavliotis2014}. \\
We now consider the probability of exit from $D$ for a solution $X(t)$ that is equal to $x$ for $t = s < T$. This probability is the solution of a boundary value problem.
\begin{theorem} Let $\mathcal L$ be the differential operator defined as \eqref{eq:LOperator}. Then
\begin{equation}\label{eq:ExitProbNotation}
	\Pr(\tau < T | X(s) = x) = \Phi(x,s,T) 
\end{equation}
where $\Phi(x,s,T)$ is the solution of the following problem
\begin{equation}\label{eq:PDEPhi}
\begin{cases}
	\frac{\partial}{\partial t} \Phi(x,t,T) + \mathcal L \Phi(x,t,T) = 0 & \text{in } D, t < T, \\
	u(x,t,T) = 1 & \text{on } \partial D, \\
	u(x,T,T) = 0 & \text{in } D.
\end{cases}
\end{equation}
\end{theorem}
Further treatment and the proof of this result can be found in \cite{Sirovich2010}. It is therefore possible to approximate $\bar\tau$ and $\Phi$ by means of classical methods for solving PDE's numerically, such as finite differences or the Finite Elements Method.

