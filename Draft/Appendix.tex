\section{Analytic expression of the mean exit time in the one-dimensional case} \label{sec:Appendix1}
In the one dimensional case, it is possible to deduce an analytic solution of \eqref{eq:PDETau}. Let us consider the domain $D = [l, r]$, the boundary condition at $x=l$ fixed as \textit{killing} and vary the boundary condition at $x=r$. Since the scope is deducing the exit time of a particle from $D$, this assumption is plausible. In this frame, it is possible to rewrite \eqref{eq:PDETau} as
\begin{equation}\label{eq:ODETau}
\left \{
\begin{aligned}
	f(x)\bar\tau'(x) + \frac{1}{2} g^2(x) \bar\tau''(x) &= -1, && l < x < r, \\
	\bar\tau(l) &= 0, \\
	\bar\tau(r) &= 0, && \text{if for $x = r$ the boundary is \textit{killing}}, \\
	\bar\tau'(r) &= 0, && \text{if for $x = r$ the boundary is \textit{reflecting}}. 
\end{aligned} \right .
\end{equation}
It is possible to show \cite{Krumscheid2015,Pavliotis2014} that $\bar\tau$ is in the one-dimensional case given by
\begin{equation}\label{eq:AnalyticTau}
	\bar\tau(x) = -2 \int_l^x \exp(-\psi(z)) \int_l^z \frac{\exp(\psi(y))}{g^2(y)}dy + c_1 \int_l^x \exp(-\psi(y))dy + c_2,
\end{equation}
where the function $\psi$ is defined as
\begin{equation}\label{eq:psi}
	\psi(x) = \int_l^x \frac{2f(y)}{g^2(y)}dy,
\end{equation}
and the constants $c_1,c_2 \in \mathbb{R}$ depend on the boundary conditions as follows
\begin{equation}\label{eq:Constants}
\begin{aligned}
	c_1 &= 2\frac{\int_l^r \exp(-\psi(z)) \int_l^z \frac{\exp(\psi(y))}{g^2(y)}dy}{\int_l^r \exp(-\psi(y))dy}, && \text{  if for $x = r$ the boundary is \textit{killing}}, \\
	c_1 &= 2\int_l^r \frac{\exp(-\psi(y))}{g(y)^2}dy, && \text{  if for $x = r$ the boundary is \textit{reflecting}}, \\
	c_2 &= 0.
\end{aligned}
\end{equation}
Let us remark that in case $f = -V'$ for some smooth function $V$ and $g = \sigma \in \mathbb{R}$, the expression of $\psi$ simplifies to
\begin{equation}\label{eq:psiSemplified}
	\psi(x) = 2\frac{V(l)-V(x)}{\sigma^2}.
\end{equation}

\clearpage
\section{Numerical approximation of the exit probability with Finite Differences in the one-dimensional case}\label{sec:Appendix2}
\renewcommand{\thepage}{\thesection -\arabic{page}}
Let us consider $D$ as the interval $\left[ l,r \right]$, the boundary condition in $l$ to be fixed to killing and in $r$ to be either killing or reflecting. In this case, since $f$ is in our case independent of $t$ and $g = \sigma \in \mathbb{R}$ \eqref{eq:PDEPhi} can be written as the following initial value PDE 
\begin{equation}\label{eq:PDEPhiOneD}
\left \{
\begin{aligned}
	-\frac{\partial}{\partial t} \Phi(t,x) + f\frac{\partial}{\partial x} \Phi(t,x) + \frac{1}{2}\sigma^2 \frac{\partial^2}{\partial x^2} \Phi(t,x) &= 0, && l < x < r \\
	\Phi(t,l) &= 1, \\
	\Phi(t,r) &= 1, && \text{if for $x = r$ the boundary is \textit{killing}} \\
	\frac{\partial}{\partial x}\Phi(t,r) &= 0, && \text{if for $x = r$ the boundary is \textit{reflecting}} \\
	\Phi(0,x) &= 0.
\end{aligned} \right .
\end{equation}
The solution of this equation can be approximated using finite differences. In particular, we employ the theta method. Let us consider the case in which $r$ is a killing boundary, $i.e.$, the PDE is endowed with pure Dirichlet boundary conditions. Given a step size $\Delta_t$ for time integration and an uniform grid $x_i = l + i\Delta_x, i=0,\dots,N+1, x_{N+1} = r$, at each timestep $k$ one has to find the solution of the linear system
\begin{equation}\label{eq:ThetaMethod}
	(I - \Delta_t\theta A) u^{k+1} = (I + \Delta_t(1-\theta) A)u^k + hF, \: 0 \leq \theta \leq 1,
\end{equation}
where $I$ is the identity matrix of $\mathbb{R}^{N\times N}$. The matrix $A$ of $\mathbb{R}^{N\times N}$ and the vector $F$ of $\mathbb{R}^N$ define the space discretization and the boundary conditions and are defined by
\begin{equation}\label{eq:ThetaMethodAandF}
	A = \frac{1}{2\Delta_x}\begin{pmatrix} 	\alpha_1 & \beta_1  &  	      &\\
						\gamma_1 & \alpha_2 & \beta_2 &\\
							 & \ddots   & \ddots  & \ddots \end{pmatrix}, \quad F = \frac{1}{2\Delta_x}\begin{pmatrix} F_1 & 0 & \cdots & 0 & F_N \end{pmatrix}^T
\end{equation}
and the coefficients are given by
\begin{equation}
\begin{split}
	\alpha_i &= -\frac{2\sigma^2}{\Delta_x}, \: i = 1, \dots, N, \\
	\beta_i  &= \frac{\sigma^2}{\Delta_x} + f(x_{i}), \: i = 1, \dots, N-1, \\
	\gamma_i &= \frac{\sigma^2}{\Delta_x} - f(x_{i+1}), \: i = 1, \dots, N-1, \\
	F_1      &= \frac{\sigma^2}{\Delta_x} - f(x_1), \\
	F_N      &= \frac{\sigma^2}{\Delta_x} - f(x_{N-1}).
\end{split}	
\end{equation}
The case of reflecting boundary condition in $x = r$ is similar and affects only the computation of the matrix $A$ and the vector $F$. In particular, we introduce a \textit{ghost node} at position $x = r + \Delta_x$, compute the derivative using a central approximation and impose that it is equal to 0, which leads to the condition that the value in the node in $x = r$ is equal to the value in $x = r - \Delta_x$. Since the matrix defining the system \eqref{eq:ThetaMethod} is tridiagonal, one can choose $\Delta_t, \Delta_x$ to be small and obtain a precise solution of \eqref{eq:PDEPhiOneD} in a reasonable computational time. Let us remark that for $\theta = 0.5$ the error of the numerical solution is of order 2 with respect to the time discretization.

 
